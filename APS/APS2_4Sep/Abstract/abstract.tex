\begin{abstracts}        


Networks have been studied  extensively  in the last decade with the purpose of  analysing  interactions  between the participating entities (nodes)  and determining the important structural patterns in such interactions.  Current research trends have  focused  on particular category of networks known as "Online
social  networks". Examples of  such networks  include social networking sites such as (Twitter, Stackoverflow, 
Facebook
,
LinkedIn), Multimedia networks (Flickr), Scientific Collaboration networks (CiteSeer), Sensor networks etc.  Networks tend to proliferate more rapidly in the Internet Era because they are no longer constrained by the geographical limitations of a conventional social network. These massive online social networks exhibit phenomena such as diffusion, contagion, influence which are the consequence of interactions seen between the entities in the network. "Latent (hidden) characteristics" is the umbrella term assigned for factors that have an influence on the activities in a social network. The focus of the present investigation is to understand the cause - effect relationship between the latent characteristics and the social network. The analysis to be carried out as a part of this study  focuses  on  the
unique issues which arise in the context of the interplay between the structural
and data-centric aspects of the network. These issues are related to understanding of the statistical properties of the network, development of techniques for community discovery, clustering of data, node classification, social influence analysis, expert location and link prediction. \\

This thesis presents an overview of the field of Social network analysis along with the important research areas in this field. The focus of the investigation is on a subset of the research areas mentioned in Chapter 1. The issues that presented in this thesis are in field of social influence analysis, community discovery and clustering of data from a network. The mathematical basis for the various approaches that are popular in the literature are presented. Secondly, simulation of the algorithms is presented on data to highlight their strengths and weaknesses of individual approaches. Based on the simulated results and surveyed literature, conclusions are inferred and future work is proposed.

\end{abstracts}
%\end{abstractlongs}


% ----------------------------------------------------------------------


%%% Local Variables: 
%%% mode: latex
%%% TeX-master: "../thesis"
%%% End: 
